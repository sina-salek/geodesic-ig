\section{Discussion}
\label{sec:discussion}

We have identified in this paper potential issues with path-based attribution methods: the presence of artefacts due to ignoring the curvatures of the model. To overcome these issues, we have introduced Geodesic Integrated Gradients, an adaptation of the original IG method which integrates gradients not along a straight line, but along the geodesic of a manifold defined by the model.

By avoiding high-gradients regions in the input space, we have shown that Geodesic IG can successfully address these issues. Moreover, it follows all of the axioms defined by \citet{sundararajan2017axiomatic}. 

We have presented two methods to approximate these geodesic paths: one based on $k$NN and the other Stochastic Variational Inference. Whilst our method shows clear advantage over the alternatives, evaluated by metrics such as Comprehensiveness and Log-Odds, there are some further research questions that we leave for future explorations. One is around computational intensiveness of our methods, as we discussed in section \ref{sec:experiments}. The other is the inherent noise in any sampling process that we may wish to use for approximating the geodesic lines. Regarding the noise, it is concievable that a method involving directly solving the geodesic equation might result in less noisy approximation of the geodesic paths. Depending on the way one solves such equations, it is possible that finding the geodesic paths might become more computationally efficient than using SVI. 